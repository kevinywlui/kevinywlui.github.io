\documentclass[addpoints]{exam}
\usepackage{amsmath}
\usepackage{amsfonts}

\newcommand{\rank}{\mathrm{rank}}
\newcommand{\nullity}{\mathrm{nullity}}
\newcommand{\spn}{\mathrm{span}}
\newcommand{\col}{\mathrm{col}}
\newcommand{\row}{\mathrm{row}}
\newcommand{\nll}{\mathrm{null}}

% \printanswers

\pagestyle{headandfoot}
\runningheadrule
\firstpageheader{}{}{}
\runningheader{Math 308L Autumn 2017}
{Midterm 2, Page \thepage\ of \numpages}
{November 15, 2017}
\firstpagefooter{}{\thepage}{}
\runningfooter{}{\thepage}{}

\begin{document}

\begin{center}
Math 308L - Autumn 2017

Midterm 2

November 15, 2017
\end{center}

\ifprintanswers
\textbf{\huge KEY}
\else
Name: \hrulefill

Student ID Number: \hrulefill
\fi

\vspace{0.3cm}

\begin{center}
    \gradetable[v][questions]
\end{center}

\vspace{0.3cm}

\begin{itemize}
    \item
        There are 5 problems on this exam. Be sure you have all 5 problems on
        your exam.
    \item
        The final answer must be left in exact form. Box your final answer.
    \item
        You are allowed the TI-30XIIS calculator. It is possible to complete
        the exam without a calculator.
    \item
        You are allowed a single sheet of 2-sided handwritten self-written notes.
    \item
        You must show your work to receive full credit. A correct answer
        with no supporting work will receive a zero.
    \item
        Use the backsides if you need extra space. Make a note of this if you
        do.
    \item
        Do not cheat. This exam should represent your own work. If you are
        caught cheating, I will report you to the Community Standards and
        Student Conduct office.
\end{itemize}

\textbf{Conventions}:
\begin{itemize}
    \item
        I will often denote the zero vector by $0$.
    \item
        When I define a variable, it is defined for that whole question. The $A$
        defined in Question 1 is the same for each part.
    \item
        I often use $x$ to denote the vector $(x_1,x_2,\ldots,x_n)$. It should be clear from context.
    \item
        Sometimes I write vectors as a row and sometimes as a column. The
        following are the same to me.
        \[
            (1,2,3) \quad
            \begin{bmatrix}
                1 \\
                2 \\
                3
            \end{bmatrix}.
        \]
    \item
        I write the evaluation of linear transforms in a few ways. The
        following are the same to me.
        \[
            T(1,2,3) \quad T((1,2,3)) \quad T \left(
            \begin{bmatrix}
                1 \\ 2\\ 3
            \end{bmatrix}
            \right)
        \]
\end{itemize}


\newpage

\begin{questions}

    \question
    Answer the following parts:
    \begin{parts}
        \part[6]
        Let
        \[
            A=
            \begin{bmatrix}
                1 & 3 & 2 \\
                0 & 1 & 1 \\
                0 & 0 & 3
            \end{bmatrix}.
        \]
        \begin{subparts}
           \subpart
            What is $A^{-1}$?
            \vfill
            \vfill
            \subpart
            What is $\det(2\cdot A^{-1})$?
            \vfill
        \end{subparts}
        \part[6]
        (Tricky.) Let
        \[
            B=
            \begin{bmatrix}
                1 & 1 & 11 \\
                -1 & 0 & 15 \\
                1 & 2 & 2017
            \end{bmatrix},
            \quad 
            y=
            \begin{bmatrix}
                1 \\ -2 \\ 0
            \end{bmatrix}.
        \]
        It turns out that $y$ is in the span of the first and second column of
        $B$ and $B$ is invertible. What is $B^{-1}y$? (Hint: Despite
        appearances, this is a quick computation.)
        \vfill
        \vfill
        \vfill
    \end{parts}

    \newpage

    \question
    Give an example of each of the following. If it is not possible, write
    ``NOT POSSIBLE''.
    \begin{parts}
        \part[3]
        Give an example of 2 linear transforms $T:\mathbb{R}^3\to \mathbb{R}^2$
        and $S:\mathbb{R}^2\to\mathbb{R}^3$ such that $T\circ
        S:\mathbb{R}^2\to\mathbb{R}^2$ is invertible.
        \vfill
        \part[3]
        Give an example of a basis for $\mathbb{R}^3$ such that every basis
        element lies in the plane $x+y+z=0$.
        \vfill
        \part[3]
        Give an example of two different matrices $A$ and $B$ such that
        $\col(A)=\col(B)$ and $\nll(A)=\nll(B)$.
        \vfill
        \part[3]
        Give an example of two $2\times 2$ matrices $A$ and $B$ such that
        $\det(A + B) \neq \det(A) + \det(B)$.
        \vfill
    \end{parts}

    \newpage

    \question
    Let $v=(1,1,-1)$ and $L_v=\spn(\{v\})$. Let $T:\mathbb{R}^3\to\mathbb{R}^3$
    be the linear transform that is the projection onto $L_v$. This tells us 2
    things about $T$:
    \begin{itemize}
        \item
            $T(x)=x$ if $x\in L_v$,
        \item
            $T(x)=0$ if $x$ is orthogonal to $v$ (so if $x\cdot v=0$).
    \end{itemize}
    There exists a matrix $A$ such that $T(x)=Ax$. The goal of this problem is
    to determine $A$.

    \begin{parts}
        \part[4]
        Give a basis for $\mathbb{R}^3$ that contains $v$ and 2 vectors
        orthogonal to $v$. (Hint: Recall that $(a_1,a_2,a_3)\cdot
        (b_1,b_2,b_3)=a_1b_1+a_2b_2+a_3b_3$.)
        \vfill
        \vfill
        \part[4]
        Answer the following questions about $A$.
        \begin{subparts}
            \subpart
            Give a basis for $\nll(A)$.
            \vfill
            \subpart
            Give a basis for $\col(A)$.
            \vfill
            \subpart
            What is the rank of $A$?
            \vfill
            \subpart
            What is $\det(A)$?
            \vfill
        \end{subparts}
        \part[4]
        What is $A$? You may express $A$ as a product of matrices and their inverses.
        \vfill
        \vfill
        \vfill
        \vfill
    \end{parts}

    \newpage

    \question 
    Let $T:\mathbb{R}^4 \to \mathbb{R}^3$ be the linear transform defined by
    $T(x)=Ax$, where $A$ and its reduced echelon form are defined as follows:
    \[
        A=
        \begin{bmatrix}
            1 & 2 & -1 & -3 \\
            2 & 4 & 0 & -4 \\
            3 & 6 & -1 & -7
        \end{bmatrix}
        \sim
        \begin{bmatrix}
            1 & 2 & 0 & -2 \\
            0 & 0 & 1 & 1 \\
            0 & 0 & 0 & 0
        \end{bmatrix}
        = B.
    \]
    To save time when writing the solutions, let's denote the columns of $A$ by
    $a_1,a_2,a_3,a_4$.
    \begin{parts}
        \part[3]
        What is a basis for $\row(A)$?
        \vfill
        \part[3]
        What a basis for the range of $T$?
        \vfill
        \part[3]
        Write the columns of $A$ corresponding to free variables as a linear
        combination of pivot columns of $A$.
        \vfill
        \part[3]
        What is a basis for $\ker(T)$?
        \vfill
    \end{parts}

    \newpage

    \question
    Let $A$ and $B$ be equivalent matrices given by
    \[
        A =
        \begin{bmatrix}
            2 & 4 & -1 & -2 \\
            -1 & -3 & -1 & 0 \\
            1 & 1 & 2 & 2 \\
            2 & 6 & 2 & 0
        \end{bmatrix}
        \sim
        \begin{bmatrix}
            1 & 0 & 0 & 1/2 \\
            0 & 1 & 0 & -1/2 \\
            0 & 0 & 1 & 1 \\
            0 & 0 & 0 & 0
        \end{bmatrix}
        = B.
    \]
    Let $a_1,a_2,a_3,a_4$ be the columns of $A$. Let $S=\spn(\{a_1,a_2\})$ and
    $T=\spn(\{a_3,a_4\})$.
    \begin{parts}
        \part[2]
        What is $\dim(\spn(\{a_1,a_2,a_3,a_4\}))$?
        \vfill
        \part[2]
        What is a basis for $\nll(A)$?
        \vfill
        \part[2]
        Denote that intersection of $S$ and $T$ by $S\cap T$. This is the
        subspace of vectors that are in $\spn(\{a_1,a_2\})$ \textbf{and} in
        $\spn(\{a_3,a_4\})$. What is $\dim(S\cap T)$?
        \vfill
        \part[6]
        (Hard.) What is a basis for $S\cap T$?
        \vfill
        \vfill
        \vfill
    \end{parts}

\end{questions}

\end{document}
