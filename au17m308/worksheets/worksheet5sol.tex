\documentclass{exam}
\usepackage{amsmath}
\usepackage{amsfonts}
\printanswers
\begin{document}

\begin{center}
    Worksheet 5 - Due 11/3
\end{center}

\begin{questions}
    \question
    Extend $\{(1,-1,0,0),(1,0,-1,0)\}$ to a basis for the subspace, $W$, defined by
    $w+x+y+z=0$. In other words, find a basis for $W$ that includes
    $(1,-1,0,0)$ and $(1,0,-1,0)$.

    \begin{solution}
        The subspace $W$ is 3-dimensional because the associated linear system
        has 3 free variables. $(1,0,0,-1)$ is not in the span of $(1,-1,0,0)$
        and $(1,0,-1,0)$ so including in the set will form a basis.
    \end{solution}

    \question
    Let $P$ be the plane given by $2x+y+z=0$ in $\mathbb{R}^3$.
    \begin{parts}
        \part
        What is a normal vector to $P$?
        \begin{solution}
            $(2,1,1)$.
        \end{solution}
        \part
        Give a basis for $\mathbb{R}^3$ that includes a normal vector to $P$
        and 2 vectors that lie on $P$.
        \begin{solution}
            $\{(2,1,1),(1,-2,0),(1,0,-2)\}$.
        \end{solution}
        \part
        Let $T:\mathbb{R}^3\to\mathbb{R}^3$ be the linear transform that
        reflects all vectors across $P$. This means that $T(n)=-n$ whenver $n$
        is normal to $P$ and $T(v)=v$ if $v$ lies on $P$. Find $A$ such that
        $T(x)=Ax$.
        \begin{solution}
            We know that
            \[
                T(2,1,1)=(-2,-1,-1),\quad
                T(1,-2,0)=(1,-2,0),\quad
                T(1,0,-2)=(1,0,-2).
            \]
            So we can use the technique described in problem 2 of worksheet 4.
            We know that the linear transform with associated matrix
            \[
                \begin{bmatrix}
                    2 & 1 & 1 \\
                    1 & -2 & 0 \\
                    1 & 0 & -2
                \end{bmatrix}^{-1}
            \]
            sends $(2,1,1)$ to $e_1$, $(1,-2,0)$ to $e_2$, and $(1,0,-2)$ to
            $e_3$. The linear transform with associated matrix
            \[
                \begin{bmatrix}
                    -2 & 1 & 1 \\
                    -1 & -2 & 0 \\
                    -1 & 0 & -2
                \end{bmatrix}
            \]
            sends $e_1$ to $(-2,-1,-1)$, $e_2$ to $(1,-2,0)$, and $e_3$ to
            $(1,0,-2)$. It follows that
            \[
                \begin{bmatrix}
                    -2 & 1 & 1 \\
                    -1 & -2 & 0 \\
                    -1 & 0 & -2
                \end{bmatrix}
                \begin{bmatrix}
                    2 & 1 & 1 \\
                    1 & -2 & 0 \\
                    1 & 0 & -2
                \end{bmatrix}^{-1} =
                \begin{bmatrix}
                    -\frac{1}{3} & -\frac{2}{3} & -\frac{2}{3} \\
                    -\frac{2}{3} & \frac{2}{3} & -\frac{1}{3} \\
                    -\frac{2}{3} & -\frac{1}{3} & \frac{2}{3}
                \end{bmatrix}
            \]
            will send $(2,1,1)$ to $(-2,-1,-1)$, $(1,-2,0)$ to $(1,-2,0)$, and
            $(1,0,-2)$ to $(1,0,-2)$.
        \end{solution}
        \part
        What is the rank of $T$? What is the nullity of $T$?
        \begin{solution}
            The associated matrix to $T$ was defined as a product of invertible
            matrices and is therefore invertible as well. This implies that $T$
            has rank 3 and nullity 0.
        \end{solution}
    \end{parts}

    \question
    Let $T:\mathbb{R}^3 \to \mathbb{R}^2$ be the linear transform defined by
    $T(1,1,1)=(1,0)$, $T(1,0,1)=(1,1)$, and $T(1,1,0)=(0,2)$.
    \begin{parts}
        \part
        Before doing a single computation, what can you already say about the
        rank and nullity of $T$?
        \begin{solution}
            It should be clear that $T$ is onto since $(1,0)$ and $(1,1)$ spans
            the codomain. Therefore, $T$ has rank 2 and nullity 1.
        \end{solution}
        \part
        Give a matrix $A$ such that $T(x)=Ax$. You may express $A$ as a product
        of matrices and their inverses.
        \begin{solution}
            This is similiar to the previous problem with the reflection. We have
            that
            \[
                A=
                \begin{bmatrix}
                    1 & 1 & 0 \\
                    0 & 1 & 2
                \end{bmatrix}
                \begin{bmatrix}
                    1 & 1 & 1 \\
                    1 & 0 & 1 \\
                    1 & 1 & 0
                \end{bmatrix}^{-1}.
            \]
        \end{solution}
        \part
        What is the rank and nullity of $T$?
        \begin{solution}
            See first part.
        \end{solution}
    \end{parts}

    \question
    Give an example of each of the following. If it is not possible, write NOT
    POSSIBLE.
    \begin{parts}
        \part
        Find an invertible $3\times 3$ matrix $A$ and a $3\times 3$ matrix $B$
        such that $rank(AB)\neq rank(BA)$.
        \begin{solution}
            NOT POSSIBLE. This is because since $A$ is invertible, we know that
            $rank(AB)=rank(BA)=rank(B)$.

            Since $A$ is invertible, it is equivalent to the identity matrix.
            This means that $A=E_1E_2\ldots E_m I_3$, where each $E_i$ is some
            row operation matrix and $I_3$ is the identity matrix. We then have
            $AB = E_1E_2\ldots E_m B$. This means that $AB$ is equivalent to
            $B$ so $rank(AB)=rank(B)$.

            To establish $rank(BA)=rank(B)$, we will take advantage of
            tranposes. We have
            \[
                rank(BA) = rank((BA)^t) = rank(A^t B^t) = rank(B^t)=rank(B).
            \]
            The first equality is because $rank(X)=rank(X^t)$ for any matrix
            $X$ because the dimension of the column space of $X$ is the same as
            the dimension of the row space of $X^t$ and these are both the
            rank. The second equality is a basic property of transposes. The
            third equality is because since $A$ is invertible, $A^t$ is also
            invertible so we can use the argument in the 2nd paragraph. The
            fourth equality is the same as the first equality.
        \end{solution}
        \part
        Find two $3\times 3$ matrices $A$ and $B$, each with nullity 1 such
        that $AB$ is the zero matrix.
        \begin{solution}
            NOT POSSIBLE. Let $T$ be the linear transform defined by $T(x)=Ax$
            and $S$ be the linear transform defined by $S(x)=Bx$. We know that
            the dimension of the range of $T$ is 2 while the dimension of the
            kernel of $S$ is 1. This means some element in the range of $T$ is
            not in the kernel of $S$. This means that $S\circ T$ is not the
            zero transformation. This means that $AB$ is not the zero matrix.
        \end{solution}
        \part
        Find two $3\times 3$ matrices $A$ and $B$, each with rank 1 such
        that $AB$ is the zero matrix.
        \begin{solution}
            Let $A$ be the matrix such that $Ax$ is the projection onto the 1st
            coordination and $B$ be the matrix such that $Bx$ is the projection
            onto the 2nd coordination. Since $AB$ is the zero matrix.
        \end{solution}
        \part
        Find two $3\times 3$ matrices $A$ and $B$, each with nullity 2 such
        that $AB$ is the zero matrix.
        \begin{solution}
            By the rank-nullity theorem, this problems in the same as part (c).
        \end{solution}
        \part
        Find two $3\times 3$ matrices $A$ and $B$, each with rank 2 such
        that $AB$ is the zero matrix.
        \begin{solution}
            By the rank-nullity theorem, this problems in the same as part (b).
        \end{solution}
    \end{parts}
\end{questions}

\end{document}
