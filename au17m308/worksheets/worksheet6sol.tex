\documentclass{exam}
\usepackage{hyperref}
\usepackage{amsmath}
\usepackage{amsfonts}

\printanswers

\begin{document}

\begin{center}
    Worksheet 6 - Due 11/10
\end{center}

\begin{questions}
    \question
    You should be familiar with the theorem here: \url{http://kevinlui.org/au17m308/log/4-review.html}
    \begin{parts}
        \part
            There is a somewhat obvious error. Find it.
            \begin{solution}
                It turns out there are (at least?) 2 errors.
                \begin{itemize}
                    \item
                        ``Let $A=[a_1,\ldots,a_m]$ be a matrix...'' The matrix
                        $A$ is of the wrong dimension.
                    \item
                        ``number of rows of all zeros in $B$.'' This is the
                        nullity of $A$ instead of the null space.
                \end{itemize}
            \end{solution}
        \part
            Write down all the different ways to express the nullity.
            \begin{solution}
                Here we use the same set up as in here: http://kevinlui.org/au17m308/log/4-review.html
                \begin{itemize}
                    \item
                        $nullity(A)$
                    \item
                        $dim(null(A))$
                    \item
                        $nullity(B)$
                    \item
                        $dim(null(B))$
                    \item
                        $dim(ker(T))$
                    \item
                        number of free variables in $B$
                    \item
                        $m-rank(A)$
                    \item
                        number of vectors required to span the solution space
                        of $Ax=0$.
                \end{itemize}
            \end{solution}
    \end{parts}

    \question
    What is the absolute value of the determinant of the matrix in problem 2 of
    worksheet 5?
    \begin{solution}
        The matrix in question defines a reflection across some plane. This
        preserves volume so the absolute value of the determinant should be 1.
    \end{solution}
\end{questions}

\end{document}
