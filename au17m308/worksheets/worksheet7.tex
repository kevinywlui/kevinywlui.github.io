\documentclass{exam}
\usepackage{hyperref}
\usepackage{amsmath}
\usepackage{amsfonts}

\begin{document}

\begin{center}
    Worksheet 7 - 11/17
\end{center}

\begin{questions}
    \question
    Is the union of two subspaces always a subspace? If not, give a counterexample.
    \question
    Is the intersection of two subspaces always a subspace? If not, give a
    counterexample.
    \question
    Let's compute a determinant using guassian elimination. Let
    \[
        A=
        \begin{bmatrix}
            2 & -1 & 2 \\
            2 & -1 & 1 \\
            0 & 3 & 1
        \end{bmatrix}
    \]
    We will determine what row operations do to the determinant of a matrix.
    \begin{parts}
        \part
        What is $\det(A)$?
        \part
        What is the effect of swapping two rows of a matrix on the determinant?
        Try swapping the first two rows of $A$ and computing the determinant of
        the resulting matrix.
        \part
        What is the effect of scale multiplying a row of a matrix on the
        determinant? Try scale multiplying the 2nd row of $A$ by 2 and
        computing the determinant of the resulting matrix.
        \part
        What is the effect of adding a multiple a row to another row on the
        determinant? Try adding twice the first row to the second and computing
        the determinant of the resulting matrix.
        \part
        Use these ideas to compute the determinant of $A$ using guassian
        elimination. If you are stuck, see the following links:
        \begin{itemize}
            \item
                \url{https://en.wikipedia.org/wiki/Gaussian_elimination#Computing_determinants}
            \item
                \url{https://math.stackexchange.com/questions/714974/determinant-by-applying-gaussian-elimination}
        \end{itemize}
    \end{parts}
\end{questions}

\end{document}
