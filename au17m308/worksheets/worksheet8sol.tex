\documentclass{exam}
\usepackage{hyperref}
\usepackage{amsmath}
\usepackage{amsfonts}
\newcommand{\rank}{\mathrm{rank}}
\newcommand{\nullity}{\mathrm{nullity}}
\newcommand{\nll}{\mathrm{null}}
\printanswers
\begin{document}

\begin{center}
    Worksheet 8 - Never due
\end{center}

\begin{questions}
    \question
    Give an example of each of the following. If it is not possible, write
    ``NOT POSSIBLE''.
    \begin{parts}
        \part
        Give an example of a basis of $\mathbb{R}^4$ such that each element
        lies in the hyperplane $2w+3x+y+z=0$.
        \begin{solution}
            NOT POSSIBLE. The hyperplane $2w+3x+y+z=0$ is a 3-dimensional
            subspace. The span of any set of vectors in a 3-dimensional
            subspace is at most 3-dimensional.
        \end{solution}
        \part
        Give an example of a basis of $\mathbb{R}^4$ such that each element
        lies in the hyperplane $2w+3x+y+z=1$.
        \begin{solution}
            Let $\{e_1,e_2,e_3,e_4\}$ be the standard basis for $\mathbb{R}^4$.
            A basis for the hyperplane is given by $\{e_1/2, e_2/3, e_3, e_4\}$.
        \end{solution}
        \part
        Give an example of a matrix that is orthogonally diagonalizable but not
        diagonalizable.
        \begin{solution}
            NOT POSSIBLE. Any orthogonally diagonalizable matrix is
            diagonalizable.
        \end{solution}
        \part
        Give an example of a matrix that is diagonalizable but not orthogonally
        diagonalizable.
        \begin{solution}
            A matrix is orthogonally diagonalizable if and only if it is
            symmetric. This problem then amounts to finding a diagonalizable
            matrix that is not symmetric. For example,
            \[
                \begin{bmatrix}
                    1 & 1 \\
                    0 & 0
                \end{bmatrix}
            \]
            This matrix is obviously not symmetric. We can see that it has
            nullity 1 so 0 is an eigenvalue. It fixes $e_1$ so 1 is an
            eigenvalue. It has 2 distinct eigenvalues so we know it is
            diagonalizable.
        \end{solution}
        \part
        Give an example of a nonzero matrix $A$ such that $A^2=0$.
        \begin{solution}
            \[
                \begin{bmatrix}
                    0 & 1 \\
                    0 & 0
                \end{bmatrix}
            \]
        \end{solution}
        \part
        Give an example of a nonzero matrix $A$ such that $A^2=I$.
        \begin{solution}
            The easier example is
            \[
                \begin{bmatrix}
                    -1
                \end{bmatrix}.
            \]
            Any reflection matrix would work.
        \end{solution}
        \part
        Give an example of a nonzero matrix $A$ such that $A^2=I$ and the
        nullity of $A$ is 1.
        \begin{solution}
            NOT POSSIBLE. If $A^2=I$ then $A^{-1}=A$ so $A$ is invertible and
            must have nullity 0.
        \end{solution}
        \part
        Give an example of an orthogonal set that is not linearly independent.
        \begin{solution}
            $\{e_1, 0\}$.
        \end{solution}
        \part
        Give an example of an orthogonal set that is not spanning.
        \begin{solution}
            $\{0\}$.
        \end{solution}
        \part
        Give an example of a $2\times 3$ matrix whose rank is equal to its
        nullity.
        \begin{solution}
            NOT POSSIBLE. Let $A$ be a matrix. By the rank-nullity
            theorem, we know that $\rank(A)+\nullity(A)=3$. Since 3 is odd, we
            know that $\rank(A)$ cannot be $\nullity(A)$.
        \end{solution}
        \part
        Give an example of 2 matrices $A$ and $B$ such that $A^3=B^3$.
        \begin{solution}
            Let $A$ be the 2d-rotation matrix by $2\pi/3$ and $B$ be the
            2d-rotation matrix by $-2\pi/3$. Then $A^3=B^3=I$.
        \end{solution}
        \part
        Give an example of 2 matrices $A$ and $B$ such that $A$ and $B$ each have
        nullity 1 but $AB$ has nullity 0.
        \begin{solution}
            NOT POSSIBLE. The nullity of $AB$ is always at least the nullity of
            $B$.
        \end{solution}
        \part
        Give an example of 2 matrices $A$ and $B$ such that $A$ and $B$ each have
        nullity 0 but $AB$ has nullity 1.
        \begin{solution}
            NOT POSSIBLE. The main idea is that the composition of two
            one-to-one functions is one-to-one.

            Suppose $A$ and $B$ are matrices with nullity 0. We will show that
            $AB$ has nullity 0 as well. Let $x\in \nll(AB)$. Then $ABx=0$. This
            means $A(Bx)=0$ so $Bx\in\nll(A)$. But $\nll(A)=\{0\}$ so $Bx=0$.
            This means $x\in \nll(B)$ but $\nll(B)=\{0\}$ so $x=0$. This proves
            that the only vectors in $\nll(AB)$ is the zero vector so the
            nullity of $AB$ is 0.
        \end{solution}
        \part
        Give an example of a diagonalizable matrix that is not invertible.
        \begin{solution}
            Any matrix with eigenvalue 0 is an example. For instance, the zero
            matrix.
        \end{solution}
        \part
        Give an example of an invertible matrix that is not diagonalizable.
        \begin{solution}
            Consider the matrix
            \[
                A=
                \begin{bmatrix}
                    1 & 1 \\
                    0 & 1
                \end{bmatrix}.
            \]
            This matrix is upper triangular so determining the eigenvalues
            amounts to reading off the diagonal. We have that $A$ has
            eigenvalue 1 with multiplicity 2. But the eigenspace corresponding
            to 1 has dimension 1. This means $A$ is not diagonalizable.

            The matrix $A$ is invertible because the determinant is 1.
        \end{solution}
        \part
        Give an example of a symmetric matrix that is not diagonalizable.
        \begin{solution}
            NOT POSSIBLE. All symmetric matrices are orthogonally
            diagonalizable and hence diagonalizable.
        \end{solution}
        \part
        Give an example of a symmetric matrix that is not invertible.
        \begin{solution}
            Zero matrix.
        \end{solution}
        \part
        Give an example of an orthogonal matrix that is not invertible.
        \begin{solution}
            NOT POSSIBLE. All orthogonal matrices, $Q$, are invertible with
            inverse $Q^t$.
        \end{solution}
        \part
        Give an example of an invertible matrix that is not orthogonal.
        \begin{solution}
            \[
                \begin{bmatrix}
                    1 & 1 \\
                    0 & 1
                \end{bmatrix}
            \]
        \end{solution}
        \part
        Give an example of a matrix with distinct eigenvalues that is not
        invertible.
        \begin{solution}
            \[
                \begin{bmatrix}
                    0 & 0 \\
                    0 & 1
                \end{bmatrix}.
            \]
        \end{solution}
        \part
        Give an example of a $3\times 3$ orthogonal matrix with only one eigenvalue.
        \begin{solution}
            The identity matrix.
        \end{solution}
        \part
        Give an example of a $3\times 3$ matrix whose only eigenvalue is 2.
        \begin{solution}
            The diagonal matrix with only 2's along the diagonal.
        \end{solution}
        \part
        Give an example of a $3\times 3$ invertible matrix whose only
        eigenvalue is 2.
        \begin{solution}
            The diagonal matrix with only 2's along the diagonal.
        \end{solution}
        \part
        Give an example of a matrix $A$ and an eigenvalue $\lambda$ such that
        the algebraic multiplicity of $\lambda$ is less than the geometric
        multiplicity.
        \begin{solution}
            NOT POSSIBLE. THe algebraic multiplicity is always at least the
            geometric multiplicity.
        \end{solution}
        \part
        Give an example of a matrix $A$ and an eigenvalue $\lambda$ such that
        the geometric multiplicity of $\lambda$ is less than the algebraic
        multiplicity.
        \begin{solution}
            Let
            \[
                A =
                \begin{bmatrix}
                    0 & 1 \\
                    0 & 0
                \end{bmatrix}.
            \]
            The eigenvalue 0 has algebraic multiplicity 2 but geometric
            multiplicity 1.
        \end{solution}
        \part
        Give an example of a matrix $A$ and an eigenvalue $\lambda$ such that
        the eigenspace is 0-dimensional.
        \begin{solution}
            NOT POSSIBLE. A number is an eigenvalue if and only if the
            corresponding eigenspace is positive-dimensional.
        \end{solution}
   \end{parts}
\end{questions}

\end{document}
