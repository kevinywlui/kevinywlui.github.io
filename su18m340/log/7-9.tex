\documentclass{article}
\usepackage{fullpage}
\usepackage{hyperref}


\begin{document}
\begin{center}
    7/9
\end{center}

\section{2.3}
\begin{enumerate}
    \item
        Recall the weird bracket thing.
    \item
        The space of all linear transformation from $V$ to $W$ is a vector
        space. It is a subspace of $F(V,W)$. It is denoted $L(V,W)$. What is
        its dimension? What is a basis for it?
    \item
        Left-shift and right-shift operations.
    \item
        The $A_{ij}$ notation is a thing.
    \item
        Let $A$ be a $m\times n$ matrix and $B$ be a $n\times p$ matrix. Then
        the product is the $m\times p$ matrix given by
        \[
            (AB)_{ij}=\sum_{k=1} ^n A_{ik} B_{kj}
        \]
        for $1\leq i\leq m$ and $1\leq j\leq p$.
    \item
        Matrix multiplication corresponds to linear map composition. See Tao's
        notes pg 99 and 100
    \item
        Given a matrix $A$, we can define the left multiplication
        transformation. Give an example.
    \item
        Give properties. Show associativity proof. See 93 in FIS.
\end{enumerate}

\section{2.4}
\begin{enumerate}
    \item 
        Isomorphism allow us to say that 2 spaces are essentially the same.
    \item
        For example, $x$-axis and $R$.
    \item
        Let $V$ and $W$ be vector spaces. Then a linear transformation is
        invertible if it has a 2-sided inverse.
    \item
        Theorem: Inverses are unique.
    \item
        Theorem: $(TU)^{-1}=U^{-1}T^{-1}$.
    \item
        Inverses are invertible.
    \item
        Bubbles. Rank is dimension of domain.
\end{enumerate}
\end{document}
