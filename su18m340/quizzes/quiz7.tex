\documentclass{article}

\usepackage{fullpage}
\usepackage{enumerate}

\begin{document}

\begin{center}
    {\bf Quiz 7}
\end{center}
Name:


\begin{enumerate}
    \item
        Prove or give a counterexample: Let $T_1:V\to V$ be a linear
        transformation with eigenvalue $\lambda_1$ and $T_2:V\to V$ be a linear
        transformation with eigenvalue $\lambda_2$. Then $\lambda_1+\lambda_2$
        is always an eigenvalue for $T_1+T_2$.
        \vfill
    \item
        Prove or give a counterexample: Let $T:V\to V$ be a linear
        transformations with eigenvalue $\lambda$. Then $\lambda^k$ is always
        an eigenvalue of $T^k$ for any positive integer $k$.
        \vfill
\end{enumerate}
    
\end{document}
