\documentclass[addpoints]{exam}
\usepackage{amsmath}
\usepackage{amsfonts}

\newcommand{\range}{\mathrm{range}}
\newcommand{\spn}{\mathrm{span}}
\newcommand{\nll}{\mathrm{null}}
\newcommand{\proj}{\mathrm{proj}}
\newcommand{\rank}{\mathrm{rank}}
\newcommand{\row}{\mathrm{row}}

\printanswers

\pagestyle{headandfoot}
\runningheadrule
\firstpageheader{}{}{}
\runningheader{Math 308L Autumn 2017}
{Final Exam, Page \thepage\ of \numpages}
{December 14, 2017}
\firstpagefooter{}{\thepage}{}
\runningfooter{}{\thepage}{}

\begin{document}

\begin{center}
Math 308L - Autumn 2017

Final Exam

December 14, 2017
\end{center}

\ifprintanswers
\textbf{\huge KEY}
\else
Name: \hrulefill

Student ID Number: \hrulefill
\fi

\vspace{0.3cm}

\begin{center}
    \gradetable[v][questions]
\end{center}

\vspace{0.3cm}

\begin{itemize}
    \item
        There are 7 problems on this exam. Be sure you have all 7 problems on
        your exam.
    \item
        The final answer must be left in exact form. Box your final answer.
    \item
        You are allowed the TI-30XIIS calculator. It is possible to complete
        the exam without a calculator.
    \item
        You are allowed a single sheet of 2-sided self-written notes.
    \item
        You must show your work to receive full credit. A correct answer
        with no supporting work will receive a zero.
    \item
        Use the backsides if you need extra space. Make a note of this if you
        do.
    \item
        Do not cheat. This exam should represent your own work. If you are
        caught cheating, I will report you to the Community Standards and
        Student Conduct office.
\end{itemize}

\textbf{Conventions}:
\begin{itemize}
    \item
        I will often denote the zero vector by $0$.
    \item
        When I define a variable, it is defined for that whole question. The $A$
        defined in Question 2 is the same for each part.
    \item
        I often use $x$ to denote the vector $(x_1,x_2,\ldots,x_n)$. It should be clear from context.
    \item
        Sometimes I write vectors as a row and sometimes as a column. The
        following are the same to me.
        \[
            (1,2,3) \quad
            \begin{bmatrix}
                1 \\
                2 \\
                3
            \end{bmatrix}.
        \]
    \item
        I write the evaluation of linear transformations in a few ways. The
        following are the same to me.
        \[
            T(1,2,3) \quad T((1,2,3)) \quad T \left(
            \begin{bmatrix}
                1 \\ 2\\ 3
            \end{bmatrix}
            \right)
        \]
\end{itemize}


\newpage

\begin{questions}


    \question
    Give an example of each of the following. If it is not possible, write
    ``NOT POSSIBLE''.
    \begin{parts}
        \part[2]
        Give an example of a $2\times 3$ matrix $A$ and a vector $b\in
        \mathbb{R}^2$ such that $Ax=b$ has no solutions but $Ax=0$ has
        infinitely many solutions.
        \begin{solution}
            Let
            \[
                A =
                \begin{bmatrix}
                    1 & 1 & 1 \\
                    0 & 0 & 0
                \end{bmatrix}
            \]
            and $b=(0,1)$.
        \end{solution}
        \vfill
        \part[2]
        Give an example of a linear system in 3 variables whose solution space
        is the intersection of the $x+y+z=0$ plane and the $xy$-plane.
        \begin{solution}
            The linear system given by
            \[
                \begin{aligned}
                    x+y+z &= 0 \\
                    z &= 0
                \end{aligned}
            \]
        \end{solution}
        \vfill
        \part[2]
        Give an example of a $2\times 2$ matrix $A$ such that $A^4 = I_2$ but
        $A^2 \neq I_2$. If possible, give the matrix $A$ explicitly.
        \begin{solution}
            Let $A$ be the rotation by $\pi/2$ matrix. This is given by
            \[
                \begin{bmatrix}
                    0 & -1 \\
                    1 & 0
                \end{bmatrix}.
            \]
        \end{solution}
        \vfill
        \part[2]
        Give an example of 2 linear transformations $T:\mathbb{R}^2\to \mathbb{R}^2$
        and $S:\mathbb{R}^2\to\mathbb{R}^2$ such that $\range(T)=\ker(S)$.
        \begin{solution}
            Let $T(x,y)=(x,y)$ and $S(x,y)=(0,0)$.
        \end{solution}
        \vfill
        \part[2]
        Give an example of an orthogonal matrix that is not invertible.
        \begin{solution}
            NOT POSSIBLE. The inverse of an orthogonal matrix is its transpose.
        \end{solution}
        \vfill
        \part[2]
        Give an example of an diagonalizable matrix that is not orthogonally
        diagonalizable.
        \begin{solution}
            \[
                \begin{bmatrix}
                    1 & 0 \\
                    1 & 0
                \end{bmatrix}
            \]
        \end{solution}
        \vfill
    \end{parts}

    \newpage

    \question
    Let $A$ be defined by
    \[
        A =
        \begin{bmatrix}
            1 & 2 & 1\\
            2 & 4 & 2\\
            0 & 0 & 1
        \end{bmatrix}.
    \]
    \begin{parts}
        \part[4]
        Find a basis for the solution space $Ax=0$.
        \begin{solution}
            $\{(2,-1,0)\}$
        \end{solution}
        \vfill
        \part[4]
        What is the general solution to $Ax=
        \begin{bmatrix}
            3 \\ 6 \\ -3
        \end{bmatrix}$?
        \begin{solution}
            $(6,0,-3)+s_1(2,-1,0)$.
        \end{solution}
        \vfill
        \part[4]
        Is there a vector $y\in \mathbb{R}^3$ such that $Ax=y$ has no
        solutions? If so, give an example. If not, why not?
        \begin{solution}
            Yes. Many possibilities.
        \end{solution}
        \vfill
    \end{parts}

    \newpage

    \question
    Let $A$ and $B$ be equivalent matrices defined by
    \[
        A =
        \begin{bmatrix}
            -3 & 3 & -1 & -9 & 3\\
            2 & -2 & 1 & 7  & -1\\
            4 & -4 & 5 & 23 & 7
        \end{bmatrix}
        \sim
        \begin{bmatrix}
            1 & -1 & 0 & 2 & -2\\
            0 & 0 & 1 & 3 & 3\\
            0 & 0 & 0 & 0 & 0
        \end{bmatrix}
        = B.
    \]
    \begin{parts}
        \part[4]
        Find a basis for the solution space of $Ax=0$.
        \begin{solution}
            $\{(1,1,0,0,0),(-2,0,-3,1,0),(2,0,-3,0,1)\}$
        \end{solution}
        \vfill
        \part[4]
        Let $a_1,a_2,a_3,a_4,a_5$ be the columns of $A$. Define $C=[a_1 \;
        a_2\; a_3 \; a_4]$. What is a particular solution to $Cx = a_5$?
        \begin{solution}
            $(-2, 0, 3, 0)$.
        \end{solution}
        \vfill
        \part[4]
        Using the same variables as (b), what is the general solution to
        $Cx=3a_4-a_5$?
        \begin{solution}
            $(8,0,6,0)+s_1(1,1,0,0)+s_2(-2,0,-3,1)$.
        \end{solution}
        \vfill
    \end{parts}

    \newpage

    \question
    Let $S$ be a subspace of $\mathbb{R}^4$ defined by
    \[
        S=
        \spn
        \left\{
            \begin{bmatrix}
                1 \\ 1 \\ 1 \\ 1
            \end{bmatrix}
            ,
            \begin{bmatrix}
                1 \\ 1 \\ 0 \\ 0
            \end{bmatrix}
            ,
            \begin{bmatrix}
                0 \\ 0 \\ 1 \\ 1
            \end{bmatrix}
            \right\}.
    \]
    \begin{parts}
        \part[3]
        What is a basis for $(S^\perp)^\perp$?
        \begin{solution}
            $\{(1,1,0,0),(0,0,1,1)\}$.
        \end{solution}
        \vfill
        \part[3]
        What is a basis for $S^\perp$?
        \begin{solution}
            $\{(1,-1,0,0),(0,0,1,-1)\}$.
        \end{solution}
        \vfill
        \part[3]
        Does there exist a rank 2 matrix $A$ such that $\nll(A)=S$? If so, give
        an example. If not, why not?
        \begin{solution}
            If $\nll(A)=S$ then $\row(A)=S^\perp$ so we can take
            \[
                \begin{bmatrix}
                    1 & -1 & 0 & 0 \\
                    0 & 0 & 1 & -1
                \end{bmatrix}.
            \]
        \end{solution}
        \vfill
        \part[3]
        Does there exist a rank 3 matrix $A$ such that $\nll(A)=S$? If so, give
        an example. If not, why not?
        \begin{solution}
            No. By the rank-nullity theorem, $\rank(A)+\nll(A)=4$. Since $\dim
            S=2$, the rank of $A$ must be 2.
        \end{solution}
        \vfill
    \end{parts}

    \newpage

    \question
    Let $T:\mathbb{R}^3\to\mathbb{R}^3$ be the linear transform defined by the
    following properties:
    \begin{itemize}
        \item
            $T(0,0,1)=(0,0,0)$,
        \item
            If $v$ is in the $xy$-plane, then $v$ is reflected across the
            $x+y=0$ plane.
    \end{itemize}
    There is a matrix $A$ such that $T(x)=Ax$. The goal of this problem is to
    understand $A$.
    \begin{parts}
        \part[3]
            Find a basis $\{u,v,w\}$ where the action of $T$ is
            well-understood. Give also $T(u), T(v)$, and $T(w)$.
            \begin{solution}
                \[
                    u = (0,0,1), T(u)=(0,0,0)
                \]
                \[
                    v = (1,1,0), T(v)=(-1,-1,0)
                \]
                \[
                    w = (1,-1,0), T(w)=(1,-1,0)
                \]
            \end{solution}
            \vfill
        \part[3]
            Find the eigenvalues of $A$ and a basis for each eigenspace of $A$.
            (Think geometrically.)
            \begin{solution}
                Part (a) gives the answer.

                $\lambda=0$ is an eigenvalue with eigenspace spanned by $u$.

                $\lambda=-1$ is an eigenvalue with eigenspace spanned by $v$.

                $\lambda=1$ is an eigenvalue with eigenspace spanned by $w$.
            \end{solution}
            \vfill
        \part[3]
            What is $A$? You may express it as product of matrices and their
            inverses.
            \begin{solution}
                Using the theory of diagonalization,
                \[
                    A =
                    \begin{bmatrix}
                        0 & 1 & 1 \\
                        0 & 1 & -1 \\
                        1 & 0 & 0
                    \end{bmatrix}
                    \begin{bmatrix}
                        0 & 0 & 0 \\
                        0 & -1 & 0 \\
                        0 & 0 & 1
                    \end{bmatrix}
                    \begin{bmatrix}
                        0 & 1 & 1 \\
                        0 & 1 & -1 \\
                        1 & 0 & 0
                    \end{bmatrix}^{-1}
                \]
            \end{solution}
            \vfill
        \part[3]
            What is $A^2$? Give it explicitly as a single matrix. (Think
            geometrically.)
            \begin{solution}
                We can see that $A^2$ is projecting onto the $xy$-plane. So
                \[
                    A^2 =
                    \begin{bmatrix}
                        1 & 0 & 0 \\
                        0 & 1 & 0 \\
                        0 & 0 & 0
                    \end{bmatrix}.
                \]
            \end{solution}
            \vfill
    \end{parts}

    \newpage

    \question
    Let $A$ be the symmetric matrix defined as
    \[
        A=
        \begin{bmatrix}
            1 & -1 & -1 \\
            -1 & 1 & -1 \\
            -1 & -1 & 1
        \end{bmatrix}
        =
        \begin{bmatrix}
            1 & -1 & -2 \\
            1 & 0 & 1 \\
            1 & 1 & 1
        \end{bmatrix}
        \begin{bmatrix}
            -1 & 0 & 0 \\
            0 & 2 & 0 \\
            0 & 0 & 2
        \end{bmatrix}
        \begin{bmatrix}
            1 & -1 & -2 \\
            1 & 0 & 1 \\
            1 & 1 & 1
        \end{bmatrix}^{-1}.
    \]
    \begin{parts}
        \part[3]
        Find the eigenvalues of $A$ and a basis for each eigenspace of $A$.
        \begin{solution}
            $\lambda = -1$ is an eigenvalue with $\{(1,1,1)\}$ as a basis for
            its eigenspace.

            $\lambda = 2$ is an eigenvalue with $\{(-1,0,1),(-2,1,1)\}$ as a
            basis for its eigenspace.
        \end{solution}
        \vfill
        \vfill
        \vfill
        \part[3]
        Find a basis for each of the following subspaces.
        \begin{itemize}
            \item
                $\nll(A)$
                \vfill
                \begin{solution}
                    Since $0$ is not an eigenvalue, $\nll(A)=\{0\}$ with basis
                    $\emptyset$.
                \end{solution}
            \item
                $\nll(A-I)$
                \begin{solution}
                    Since $1$ is not an eigenvalue, $\nll(A)=\{0\}$ with basis
                    $\emptyset$.
                \end{solution}
                \vfill
            \item
                $\nll(A-2I)$.
                \begin{solution}
                    We have that $\nll(A-2I)=E_2$ which has basis
                    $\{(-1,0,1),(-2,1,1)\}$.
                \end{solution}
                \vfill
        \end{itemize}
        \vfill
        \part[3]
        Find an orthogonal matrix $Q$ and a diagonal matrix $D$ such that
        $A=QDQ^{-1}$.
        \begin{solution}
            We use Gram-Schmidt to perform an orthogonal basis for each
            eigenspace.

            An orthonormal basis for the eigenspace corresponding to
            $\lambda=-1$ is $\{(1/3,1/3,1/3)\}$.

            An orthonormal basis for $\lambda=2$ is
            $\{\frac{1}{\sqrt{2}}(-1,0,1),\sqrt{\frac{2}{3}}(-1/2,1,-1/2)\}$.
        \end{solution}
        \vfill
        \vfill
        \vfill
        \part[3]
        Find all $k\in \mathbb{R}$ such that $A-kI_3$ is not invertible.
        \begin{solution}
            $k=-1,2$.
        \end{solution}
        \vfill
    \end{parts}


    \newpage

    \question
    Let $v=(2,2,1)$ and $T:\mathbb{R}^3\to\mathbb{R}^3$ be defined by
    $T(x)=\proj_v x$.
    \begin{parts}
        \part[4]
        Find an orthogonal basis for $\mathbb{R}^3$ that contains $v$. (Hint:
        first find a basis for $\mathbb{R}^3$ that contains $v$.)
        \begin{solution}
            $\{(2,2,1),(1,0,-2),(0,1,-2)\}$.
        \end{solution}
        \vfill
        \vfill
        \vfill
        \vfill
        \vfill
        \vfill
        \vfill
        \part[4]
        There exists a matrix $A$ such that $T(x)=Ax$. Find the eigenvalues of
        $A$ and a basis for each eigenspace of $A$. (Hint: see part (a).)
        \begin{solution}
            The eigenspace corresponding to $1$ is spanned by $(2,2,1)$.

            The eigenspace corresponding to $0$ is spanned by
            $(1,0,-2),(0,1,-2)$.
        \end{solution}
        \vfill
        \vfill
        \vfill
        \vfill
        \part[4]
        Let $e_1=(1,0,0)$. Evaluate the following:
        \begin{itemize}
            \item
                $Ae_1$
                \begin{solution}
                    This is $T(e_1)=\proj_{v} e_1 = (4/9,4/9,2/9)$.
                \end{solution}
                \vfill
            \item
                $A^2e_1$
                \begin{solution}
                   Doing two projections is the same as one.
                \end{solution}
                \vfill
            \item
                $A^{100}e_1$
                \begin{solution}
                    Doing one hundred projections is the same as one.
                \end{solution}
                \vfill
        \end{itemize}
        \vfill
    \end{parts}


\end{questions}

\end{document}
