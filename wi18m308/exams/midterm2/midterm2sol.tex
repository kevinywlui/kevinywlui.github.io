\documentclass[addpoints]{exam}
\usepackage{amsmath}
\usepackage{amsfonts}

\makeatletter
\renewcommand*\env@matrix[1][*\c@MaxMatrixCols c]{%
  \hskip -\arraycolsep
  \let\@ifnextchar\new@ifnextchar
  \array{#1}}
\makeatother

\newcommand{\rank}{\mathrm{rank}}
\newcommand{\nullity}{\mathrm{nullity}}
\newcommand{\spn}{\mathrm{span}}
\newcommand{\col}{\mathrm{col}}
\newcommand{\row}{\mathrm{row}}
\newcommand{\nll}{\mathrm{null}}
\newcommand{\range}{\mathrm{range}}

\printanswers

\pagestyle{headandfoot}
\runningheadrule
\firstpageheader{}{}{}
\runningheader{Math 308L Autumn 2017}
{Midterm 2, Page \thepage\ of \numpages}
{November 15, 2017}
\firstpagefooter{}{\thepage}{}
\runningfooter{}{\thepage}{}

\begin{document}

\begin{center}
Math 308L - Autumn 2017

Midterm 2

November 15, 2017
\end{center}

\ifprintanswers
\textbf{\huge KEY}
\else
Name: \hrulefill

Student ID Number: \hrulefill
\fi

\vspace{0.3cm}

\begin{center}
    \gradetable[v][questions]
\end{center}

\vspace{0.3cm}

\begin{itemize}
    \item
        There are 5 problems on this exam. Be sure you have all 5 problems on
        your exam.
    \item
        The final answer must be left in exact form. Box your final answer.
    \item
        You are allowed the TI-30XIIS calculator. It is possible to complete
        the exam without a calculator.
    \item
        You are allowed a single sheet of 2-sided handwritten self-written notes.
    \item
        You must show your work to receive full credit. A correct answer
        with no supporting work will receive a zero.
    \item
        Use the backsides if you need extra space. Make a note of this if you
        do.
    \item
        Do not cheat. This exam should represent your own work. If you are
        caught cheating, I will report you to the Community Standards and
        Student Conduct office.
\end{itemize}

\textbf{Conventions}:
\begin{itemize}
    \item
        I will often denote the zero vector by $0$.
    \item
        When I define a variable, it is defined for that whole question. The $A$
        defined in Question 1 is the same for each part.
    \item
        I often use $x$ to denote the vector $(x_1,x_2,\ldots,x_n)$. It should be clear from context.
    \item
        Sometimes I write vectors as a row and sometimes as a column. The
        following are the same to me.
        \[
            (1,2,3) \quad
            \begin{bmatrix}
                1 \\
                2 \\
                3
            \end{bmatrix}.
        \]
    \item
        I write the evaluation of linear transforms in a few ways. The
        following are the same to me.
        \[
            T(1,2,3) \quad T((1,2,3)) \quad T \left(
            \begin{bmatrix}
                1 \\ 2\\ 3
            \end{bmatrix}
            \right)
        \]
\end{itemize}


\newpage

\begin{questions}

    \question
    Answer the following parts:
    \begin{parts}
        \part[6]
        Let
        \[
            A=
            \begin{bmatrix}
                1 & 3 & 2 \\
                0 & 1 & 1 \\
                0 & 0 & 3
            \end{bmatrix}.
        \]
        \begin{subparts}
           \subpart
            What is $A^{-1}$?
            \begin{solution}
                \[
                    A^{-1}=
                    \begin{bmatrix}
                        1 & -3 & 1/3 \\
                        0 & 1 & -1/3 \\
                        0 & 0 & 1/3
                    \end{bmatrix}
                \]
            \end{solution}
            \vfill
            \vfill
            \subpart
            What is $\det(2\cdot A^{-1})$?
            \begin{solution}
                \[
                    \det(2\cdot A^{-1})=2^3 \det(A^{-1}) = 8/3.
                \]
            \end{solution}
            \vfill
        \end{subparts}
        \part[6]
        (Tricky.) Let
        \[
            B=
            \begin{bmatrix}
                1 & 1 & 11 \\
                -1 & 0 & 15 \\
                1 & 2 & 2017
            \end{bmatrix},
            \quad
            y=
            \begin{bmatrix}
                1 \\ -2 \\ 0
            \end{bmatrix}.
        \]
        It turns out that $y$ is in the span of the first and second column of
        $B$ and $B$ is invertible. What is $B^{-1}y$? (Hint: Despite
        appearances, this is a quick computation.)
        \begin{solution}
            Denote the columns of $B$ by $b_1,b_2,b_3$. Let $B^{-1}y=x$. Then
            $Bx=y$ and we are trying to find $x$. This amounts to solving a
            linear system. If $x=(x_1,x_2,x_3)$, then
            \[
                x_1b_1+x_2b_2+x_3b_3 = y.
            \]
            But we know that $y$ is in the span of $b_1$ and $b_2$ so $x_3=0$
            and we are left with
            \[
                x_1b_1+x_2b_2 = y.
            \]
           This yields the much easier linear system
            \[
                \begin{bmatrix}[cc|c]
                    1 & 1 & 1 \\
                    -1 & 0 & -2 \\
                    1 & 2 & 0
                \end{bmatrix}
            \]
            which has solutions $x_1=2$ and $x_2=-1$. So altogether, we have
            \[
                (x_1,x_2,x_3)=(2,-1,0).
            \]
        \end{solution}
        \vfill
        \vfill
        \vfill
    \end{parts}

    \newpage

    \question
    Give an example of each of the following. If it is not possible, write
    ``NOT POSSIBLE''.
    \begin{parts}
        \part[3]
        Give an example of 2 linear transforms $T:\mathbb{R}^3\to \mathbb{R}^2$
        and $S:\mathbb{R}^2\to\mathbb{R}^3$ such that $T\circ
        S:\mathbb{R}^2\to\mathbb{R}^2$ is invertible.
        \begin{solution}
            Let $T(x,y,z)=(x,y)$ and $S(x,y)=(x,y,0)$. Then $(T\circ
            S)(x,y)=(x,y)$ which is the identity transform which is invertible.
        \end{solution}
        \vfill
        \part[3]
        Give an example of a basis for $\mathbb{R}^3$ such that every basis
        element lies in the plane $x+y+z=0$.
        \begin{solution}
            NOT POSSIBLE. The set $x+y+z=0$ is a 2-dimensional subspace. There
            is no basis of $\mathbb{R}^3$ that lie in a 2-dimensional subspace.
        \end{solution}
        \vfill
        \part[3]
        Give an example of two different matrices $A$ and $B$ such that
        $\col(A)=\col(B)$ and $\nll(A)=\nll(B)$.
        \begin{solution}
            Pick your favorite natural number $n$. Let $A=I_n$ and $B=2\cdot
            I_n$. Then $\col(A)=\col(B)=\mathbb{R}^n$ and
            $\nll(A)=\nll(B)=\{0\}$.
        \end{solution}
        \vfill
        \part[3]
        Give an example of two $2\times 2$ matrices $A$ and $B$ such that
        $\det(A + B) \neq \det(A) + \det(B)$.
        \begin{solution}
            Let $A=I_2$ and $B=I_2$. Then
            \[
                \det(A+B)=4 \quad \text{and} \quad \det(A)=\det(B)=1.
            \]
        \end{solution}
        \vfill
    \end{parts}

    \newpage

    \question
    Let $v=(1,1,-1)$ and $L_v=\spn(\{v\})$. Let $T:\mathbb{R}^3\to\mathbb{R}^3$
    be the linear transform that is the projection onto $L_v$. This tells us 2
    things about $T$:
    \begin{itemize}
        \item
            $T(x)=x$ if $x\in L_v$,
        \item
            $T(x)=0$ if $x$ is orthogonal to $v$ (so if $x\cdot v=0$).
    \end{itemize}
    There exists a matrix $A$ such that $T(x)=Ax$. The goal of this problem is
    to determine $A$.

    \begin{parts}
        \part[4]
        Give a basis for $\mathbb{R}^3$ that contains $v$ and 2 vectors
        orthogonal to $v$. (Hint: Recall that $(a_1,a_2,a_3)\cdot
        (b_1,b_2,b_3)=a_1b_1+a_2b_2+a_3b_3$.)
        \begin{solution}
            There are two methods to find 2 vectors orthogonal to $v$.
            \begin{itemize}
                \item
                    Eyeball it.
                \item
                    Find 2 particular solutions to $v\cdot
                    (x,y,z)=x+y-z=0$.
            \end{itemize}
            We find that $(1,0,1)$ and $(0,1,1)$ are distinct vector orthogonal
            to $v$. A basis for $\mathbb{R}^3$ is then given by
            \[
                \{v,(1,0,1),(0,1,1)\}.
            \]
        \end{solution}
        \vfill
        \vfill
        \part[4]
        Answer the following questions about $A$.
        \begin{subparts}
            \subpart
            Give a basis for $\nll(A)$.
            \begin{solution}
                The null space of $A$ is the kernel of $T$. We see that this is
                spanned by $\{(1,0,1),(0,1,1)\}$.
            \end{solution}
            \vfill
            \subpart
            Give a basis for $\col(A)$.
            \begin{solution}
                We can see that $T$ sends everything to $L_v$ so a basis for
                $\col(A)=\range(T)$ is $\{v\}$.
            \end{solution}
            \vfill
            \subpart
            What is the rank of $A$?
            \begin{solution}
                From the last part, we can see that the rank is 1.
            \end{solution}
            \vfill
            \subpart
            What is $\det(A)$?
            \begin{solution}
                The determinant of $A$ is zero as $T$ is not invertible.
            \end{solution}
            \vfill
        \end{subparts}
        \part[4]
        What is $A$? You may express $A$ as a product of matrices and their inverses.
        \begin{solution}
            From past worksheets and lectures, we know that if
            $\{u_1,\ldots,u_n\}$ is a basis for $\mathbb{R}^n$ and
            $T:\mathbb{R}^n\to\mathbb{R}^n$ is defined so that $T(u_i)=v_i$ for
            $i=1,\ldots,n$ then the corresponding matrix for $T$ is given by
            $VU^{-1}$, where $V=[v_i]$ and $U=[u_i]$.

            In this case, we have that $T(v)=v$ and $T(1,0,1)=(0,0,0)$ and
            $T(0,1,1)=(0,0,0)$. The corresponding matrix is
            \[
                A=
                \begin{bmatrix}
                    1 & 0 & 0 \\
                    1 & 0 & 0 \\
                    -1 & 0 & 0
                \end{bmatrix}
                \begin{bmatrix}
                    1 & 1 & 0 \\
                    1 & 0 & 1 \\
                    -1 & 1 & 1
                \end{bmatrix}^{-1}.
            \]
        \end{solution}
        \vfill
        \vfill
        \vfill
        \vfill
    \end{parts}

    \newpage

    \question
    Let $T:\mathbb{R}^4 \to \mathbb{R}^3$ be the linear transform defined by
    $T(x)=Ax$, where $A$ and its reduced echelon form are defined as follows:
    \[
        A=
        \begin{bmatrix}
            1 & 2 & -1 & -3 \\
            2 & 4 & 0 & -4 \\
            3 & 6 & -1 & -7
        \end{bmatrix}
        \sim
        \begin{bmatrix}
            1 & 2 & 0 & -2 \\
            0 & 0 & 1 & 1 \\
            0 & 0 & 0 & 0
        \end{bmatrix}
        = B.
    \]
    To save time when writing the solutions, let's denote the columns of $A$ by
    $a_1,a_2,a_3,a_4$.
    \begin{parts}
        \part[3]
        What is a basis for $\row(A)$?
        \begin{solution}
            A basis for $\row(A)$ is given by the 2 nonzero rows of $B$.
        \end{solution}
        \vfill
        \part[3]
        What a basis for the range of $T$?
        \begin{solution}
            A basis for the range is given by $\{a_1,a_3\}$.
        \end{solution}
        \vfill
        \part[3]
        Write the columns of $A$ corresponding to free variables as a linear
        combination of pivot columns of $A$.
        \begin{solution}
            Relations among the columns of $A$ are exactly the relations among
            the columns of $B$. This means
            \[
                a_2 = 2 a_1 \quad \text{and} \quad a_4 = -2a_1 + a_3.
            \]
        \end{solution}
        \vfill
        \part[3]
        What is a basis for $\ker(T)$?
        \begin{solution}
            The general solution to $A$ is $x=s_1(-2,1,0,0)+s_2(2,0,-1,1)$.
            This means that a basis for $\ker(T)$ is
            $\{(-2,1,0,0),(2,0,-1,1)\}$.
        \end{solution}
        \vfill
    \end{parts}

    \newpage

    \question
    Let $A$ and $B$ be equivalent matrices given by
    \[
        A =
        \begin{bmatrix}
            2 & 4 & -1 & -2 \\
            -1 & -3 & -1 & 0 \\
            1 & 1 & 2 & 2 \\
            2 & 6 & 2 & 0
        \end{bmatrix}
        \sim
        \begin{bmatrix}
            1 & 0 & 0 & 1/2 \\
            0 & 1 & 0 & -1/2 \\
            0 & 0 & 1 & 1 \\
            0 & 0 & 0 & 0
        \end{bmatrix}
        = B.
    \]
    Let $a_1,a_2,a_3,a_4$ be the columns of $A$. Let $S=\spn(\{a_1,a_2\})$ and
    $T=\spn(\{a_3,a_4\})$.
    \begin{parts}
        \part[2]
        What is $\dim(\spn(\{a_1,a_2,a_3,a_4\}))$?
        \begin{solution}
            This is asking for the rank of $A$ which is 3 because there are 3
            nonzero rows of $B$.
        \end{solution}
        \vfill
        \part[2]
        What is a basis for $\nll(A)$?
        \begin{solution}
            The general solution to $A$ is $x=s_1(-1/2,1/2,-1,1)$. This means
            that a basis for $\nll(A)$ is $\{(-1/2,1/2,-1,1)\}$.
        \end{solution}
        \vfill
        \part[2]
        Denote that intersection of $S$ and $T$ by $S\cap T$. This is the
        subspace of vectors that are in $\spn(\{a_1,a_2\})$ \textbf{and} in
        $\spn(\{a_3,a_4\})$. What is $\dim(S\cap T)$?
        \begin{solution}
            We can see that $S$ and $T$ are distinct 2-dimensional spaces in
            $\col(A)$ which is a 3-dimensional space. This means that the
            intersection is 1-dimensional. This is the more geometric idea. See
            the next answer for the more algebraic one.
        \end{solution}
        \vfill
        \part[6]
        (Hard.) What is a basis for $S\cap T$?
        \begin{solution}
            We will investigate what it means for a vector to be in $S\cap T$.
            Suppose $v\in S\cap T$. This means that we can write $v$ as a
            linear combination of $a_1,a_2$ and as a linear combination of
            $a_3,a_4$. So there exists scalar $c_1,c_2,c_3,c_4$ such that
            \[
                v=c_1a_1+c_2a_2=c_3a_3+c_4a_4.
            \]
            The goal is to determine the constraints on $c_1,c_2,c_3,c_4$. By
            rearranging, this means that
            \[
                c_1a_1+c_2a_2 - c_3a_3 - c_4a_4=0.
            \]
            This means that $(c_1,c_2,-c_3,-c_4)\in \nll(A)$. Then
            $(c_1,c_2,-c_3,-c_4)=s_1(-1/2,1/2,-1,1)$ for some $s_1$. So
            \begin{equation}
                \label{a}
                v=-1/2s_1a_1+1/2s_1a_2
            \end{equation}
            for some $s_1$. The equation \eqref{a} characterizes all $v\in S\cap
            T$. We can see that it is a 1-dimensional space. A basis is
            obtained by plugging in any nonzero $s_1$ into \eqref{a} so a basis
            for $S\cap T$ is $\{a_1-a_2\}$.
        \end{solution}
        \vfill
        \vfill
        \vfill
    \end{parts}

\end{questions}

\end{document}
