\documentclass[addpoints]{exam}
\usepackage{amsmath}
\usepackage{amsfonts}
\usepackage{multicol}
\newcommand{\col}{\mathrm{col}}
\newcommand{\nll}{\mathrm{null}}
\newcommand{\row}{\mathrm{row}}
\newcommand{\spn}{\mathrm{span}}
\newcommand{\rank}{\mathrm{rank}}
\newcommand{\nullity}{\mathrm{nullity}}
\newcommand{\range}{\mathrm{range}}

\printanswers

\pagestyle{headandfoot}
\runningheadrule
\firstpageheader{}{}{}
\runningheader{Math 308H Winter 2018}
{Final, Page \thepage\ of \numpages}
{2018-03-15}
\firstpagefooter{}{\thepage}{}
\runningfooter{}{\thepage}{}

\begin{document}

\begin{center}
    Math 308H - Winter 2018

    Final

    2018-03-15
\end{center}

\ifprintanswers
\textbf{\huge KEY}
\else
Name: \hrulefill

Student ID Number: \hrulefill
\fi

\vspace{0.3cm}

\begin{center}
    \gradetable[v][questions]
\end{center}

\vspace{0.3cm}

\begin{itemize}
    \item
        There are 6 problems on this exam. Be sure you have all 6 problems on
        your exam.
    \item
        The final answer must be left in exact form. Box your final answer.
    \item
        You are allowed the TI-30XIIS calculator. It is possible to complete
        the exam without a calculator.
    \item
        You are allowed a single sheet of 2-sided handwritten self-written notes.
    \item
        You must show your work to receive full credit. A correct answer
        with no supporting work will receive a zero.
    \item
        Use the backsides if you need extra space. Make a note of this if you
        do.
    \item
        Do not cheat. This exam should represent your own work. If you are
        caught cheating, I will report you to the Community Standards and
        Student Conduct office.
\end{itemize}

\textbf{Conventions}:
\begin{itemize}
    \item
        I will often denote the zero vector by $0$.
    \item
        When I define a variable, it is defined for that whole question. The $A$
        defined in Question 2 is the same for each part.
    \item
        I treat row and column vectors as the same.
    \item
        For any linear transformation $T$, there exists a matrix $A$ such that
        $T(x)=Ax$. I defined the determinant, rank, and nullity of $T$ using
        $A$. This means,
        \[
            \det(T)=\det(A), \quad \rank(T)=\rank(A), \quad \nullity(T)=\nullity(A).
        \]
\end{itemize}


\newpage

\begin{questions}

    % Question 1
    \question
    Give an example of each of the following. If it is not possible, write ``NOT
    POSSIBLE''. You do not need to justify your answers.
    \begin{parts}
        \part[2]
        If possible, give an example of a linear system of equations whose
        solution space is the $(1,2,3)+s_1(1,0,0)$ line.
        \begin{solution}
            \[
                y=2,\; z=3
            \]
        \end{solution}
        \vfill
        \part[2]
        If possible, give an example of a $2\times 2$ matrix $A$ such that
        $A\neq 0,I$ and $A(A-I)=0$.
        \begin{solution}
            This means that $A^2=A$ so any projection matrix would work. For
            example,
            \[
                A =
                \begin{bmatrix}
                    1 & 0 \\
                    0 & 0
                \end{bmatrix}.
            \]
        \end{solution}
        \vfill
        \part[2]
        If possible, give an example of a $2\times 2$ invertible matrix, $A$,
        such that $e_1-e_2\notin \col(A)$.
        \begin{solution}
            NOT POSSIBLE. An invertible matrix must have spanning columns.
        \end{solution}
        \vfill
        \part[2]
        If possible, give an example of two invertible $2\times 2$ matrices $A$
        and $B$ such that $A+B$ is not invertible.
        \begin{solution}
            Let $A=-B=I$.
        \end{solution}
        \vfill
        \part[2]
        If possible, give an example of two $2\times 2$ matrices $A$ and $B$
        that are neither zero nor the identity matrix such that $AB=BA$.
        \vfill
        \begin{solution}
            Take any two diagonal matrices that are not zero or the identity.
        \end{solution}
        \part[2]
        If possible, give an example of two linear transformation
        $T:\mathbb{R}^2\to\mathbb{R}^2$ and $S:\mathbb{R}^2\to\mathbb{R}^2$
        such that $2$ is an eigenvalue of $T$ and $3$ is an eigenvalue of $S$
        but $6$ is not an eigenvalue of $T\circ S$.
        \begin{solution}
            $T(x,y)=(2x,0)$, $S(x,y)=(0,3y)$.
        \end{solution}
        \vfill
    \end{parts}

    \newpage
    % Question 2
    \question
    \begin{parts}
        \part[6]
        Let
        \[
            A = 
            \begin{bmatrix}
                1 & -3 \\
                0 & 2
            \end{bmatrix}.
        \]
        \begin{enumerate}
            \item
                What is the characteristic polynomial of $A^{-1}$? 
                \begin{solution}
                    $(1-\lambda)(1/2-\lambda)$.
                \end{solution}
                \vfill
            \item
                The matrix $A$ is diagonalizable so it can be written as
                $A=UDU^{-1}$. What is $U$ and $D$?
                \begin{solution}
                    \[
                        U =
                        \begin{bmatrix}
                            1 & 3 \\
                            0 & -1
                        \end{bmatrix}
                        \quad
                        D = 
                        \begin{bmatrix}
                            1 & 0 \\
                            0 & 2
                        \end{bmatrix}
                    \]
                \end{solution}
                \vfill
                \vfill
                \vfill
        \end{enumerate}
        \part[6]
        Let
        \[
            B =
            \begin{bmatrix}
                4 & 0 & 2 \\
                0 & 1 & 2 \\
                0 & 2 & 4
            \end{bmatrix}.
        \]
        \begin{enumerate}
            \item 
                What is the reduced echelon form of $B$?
                \begin{solution}
                    \[
                        \begin{bmatrix}
                            1 & 0 & 1/2 \\
                            0 & 1 & 2 \\
                            0 & 0 & 0
                        \end{bmatrix}
                    \]
                \end{solution}
                \vfill
                \vfill
            \item
                What is the general solution to $Bx=(6,3,6)$?
                \begin{solution}
                    \[
                        (1,1,1)+s_1(-1/2,-2,1).
                    \]
                \end{solution}
                \vfill
                \vfill
        \end{enumerate}
    \end{parts}

    \newpage
    % Question 3
    \question
    
    Let $A$ and $B$ be equivalent matrices defined by
    \[
        A = 
        \begin{bmatrix}
            1 & 2 & 0 & -1 \\
            0 & 1 & 0 & 3 \\
            0 & 1 & 1 & 0 \\
            2 & 5 & 0 & 1
        \end{bmatrix}
        \sim
        \begin{bmatrix}
            1 & 0 & 0 & -7 \\
            0 & 1 & 0 & 3 \\
            0 & 0 & 1 & -3 \\
            0 & 0 & 0 & 0
        \end{bmatrix}
        =B 
    \]
    Let $a_1,a_2,a_3,a_4$ denote the columns of $A$.
    \begin{parts}
        \part[3]
        Do not write express a basis as a matrix.
        \begin{enumerate}
            \item 
                Give a basis for $\col(2A^t)$.
                \begin{solution}
                    The first thing to note that is $\col(2A^t)=\row(A)$. A
                    basis is then
                    \[
                        \{(1,0,0,-7),(0,1,0,3),(0,0,1,-3)\}
                    \]
                \end{solution}
                \vfill
            \item 
                Give a basis for $\nll(A)$.
                \begin{solution}
                    \[
                        \{(7,-3,3,1)\}
                    \]
                \end{solution}
                \vfill
            \item
                Give a basis for $\row(A)$.
                \begin{solution}
                    \[
                        \{(1,0,0,-7),(0,1,0,3),(0,0,1,-3)\}
                    \]
                \end{solution}
                \vfill
        \end{enumerate}
        \part[3]
        These should be quick questions.
        \begin{enumerate}
            \item
                What is $\rank(A)$?
                \begin{solution}
                    3
                \end{solution}
                \vfill
            \item
                What is $\nullity(A^{t} D^{-1})$, where $D$ is the $4\times 4$ diagonal
                matrix consisting of $1,2,3,4$ along the diagonal.
                \begin{solution}
                    1
                \end{solution}
                \vfill
            \item
                What is $\det(2A)$?
                \begin{solution}
                    0
                \end{solution}
                \vfill
        \end{enumerate}
        \part[3]
        Give a nontrivial linear combination of the columns of $A$ that sum to
        zero. You may use $a_1,a_2,a_3,a_4$ to denote the columns of $A$.
        \begin{solution}
            $7a_1-3a_2+3a_3+a_4=0$.
        \end{solution}
        \vfill
        \vfill
        \part[3]
        Let $C$ be the $4\times 3$ matrix given by $C = [a_1 \; a_2 \; a_3]$.
        So $C$ is the submatrix of $A$ consisting of the first 3 columns. Give
        the general solution for $Cx=a_1+a_4$.
        \begin{solution}
            From the previous part, we know that $a_4=-7a_1+3a_2-3a_3$. This
            means that $a_1+a_4=-6a_1+3a_2-3a_3$. The general solution is then
            \[
                x=(-6,3,-3).
            \]
            There is no homogenous part because the columns of $C$ are linearly
            independent.
        \end{solution}
        \vfill
        \vfill
    \end{parts}

    \newpage
    % Question 4
    \question
    Let $T:\mathbb{R}^4\to\mathbb{R}^3$ be the linear transformation defined by
    \[
        T(w,x,y,z)=(w+y+z,x+y+z,x+y+z).
    \]
    \begin{parts}
        \part[3]
        There is a matrix $A$ such that $T(x)=Ax$. What is $A$?
        \begin{solution}
            \[
                A =
                \begin{bmatrix}
                    1 & 0 & 1 & 1\\
                    0 & 1 & 1 & 1\\
                    0 & 1 & 1 & 1
                \end{bmatrix}
            \]
        \end{solution}
        \vfill
        \part[3]
        Let $v=(0,3,0,8)$. Give the general solution to $Ax=2Av+(2,1,1)$.
        \begin{solution}
            A particular solution to $Ax=2Av$ is $x=2v$. A particular solution
            to $Ax=(2,1,1)$ is $(2,1,0)$. The general solution to the
            homogenous system $Ax=0$ is $s_1(-1,-1,1,0)+s_2(-1,-1,0,1)$. The
            general solutin to $Ax=2Av=(2,1,1)$ is then
            \[
                2v+(2,1,0)+s_1(-1,-1,1,0)+s_2(-1,-1,0,1).
            \]
        \end{solution}
        \vfill
        \vfill
        \part[3] 
        Does there exists a rank 2 linear transformation $S$ such that $T\circ
        S$ is the zero transformation? If so, give an example. If not, why not?
        \begin{solution}
            Yes. If $T\circ S=0$ then $\range(S)\subseteq \ker(T)$. We know a
            basis for $\ker(T)$ so define
            \[
                S(x) =
                \begin{bmatrix}
                    -1 & -1 \\
                    -1 & -1 \\
                    1 & 0 \\
                    0 & 1
                \end{bmatrix}x.
            \]
        \end{solution}
        \vfill
        \part[3] 
        Does there exists a rank 3 linear transformation $S$ such that $T\circ
        S$ is the zero transformation? If so, give an example. If not, why not?
        \begin{solution}
            No. If $\range(S)\subseteq \ker(T)$, then $\rank(S)\leq
            \nullity(T)$.
        \end{solution}
        \vfill
    \end{parts}

    \newpage
    % Question 5
    \question
    Let
    \[
        A =
        \begin{bmatrix}
            0 & -1 & \frac{37}{3} & -\frac{253}{15} \\
            0 & 2 & 0 & -\frac{1}{5} \\
            0 & 0 & 2 & \frac{7}{5} \\
            0 & 0 & 0 & 3
        \end{bmatrix}
    \]
    be a matrix which decomposes as $A=UDU^{-1}$, where
    \[
        U = 
        \begin{bmatrix}
            1 & -1 & 18 & 1 \\
            0 & 2 & 1 & -1 \\
            0 & 0 & 3 & 7 \\
            0 & 0 & 0 & 5
        \end{bmatrix},
        \quad 
        D =
        \begin{bmatrix}
            0 & 0 & 0 & 0 \\
            0 & 2 & 0 & 0 \\
            0 & 0 & 2 & 0 \\
            0 & 0 & 0 & 3
        \end{bmatrix}.
    \]
    Let $u_1,u_2,u_3,u_4$ be the columns of $U$ and
    $\mathcal{B}=\{u_1,u_2,u_3,u_4\}$.
    \begin{parts}
        \part[6]
        Fill out this table.
        \ifprintanswers
        \begin{center}
            \begin{tabular}{|l|l|l|l|}
                \hline
                Eigenvalue $\lambda$ & Alg. Multiplicity of $\lambda$ &
                Geo. Multiplicity of $\lambda$ & Basis for $E_\lambda$ \\
                \hline 
                 0 & 1 & 1 & $\{u_1\}$\\[10ex]
                 \hline
                 2 & 2 & 2 & $\{u_2,u_3\}$\\[10ex]
                 \hline
                 3 & 1 & 1 & $\{u_4\}$\\[10ex]
                 \hline
            \end{tabular}
        \end{center}
        \else
        \begin{center}
            \begin{tabular}{|l|l|l|l|}
                \hline
                Eigenvalue $\lambda$ & Alg. Multiplicity of $\lambda$ &
                Geo. Multiplicity of $\lambda$ & Basis for $E_\lambda$ \\
                \hline 
                 & & & \\[10ex]
                 \hline
                 & & & \\[10ex]
                 \hline
                 & & & \\[10ex]
                 \hline
            \end{tabular}
        \end{center}
        \fi
        \part[3]
        Let $x=u_1+u_2+u_3+u_4$. Express $A^{18}x$ as a linear combination of
        $u_1,u_2,u_3,u_4$. You are allowed to have exponents of numbers in your
        answer. (Hint: $x$ has been expressed as the sum of eigenvectors.)
        \begin{solution}
            $2^{18}u_2+2^{18}u_3+3^{18}u_4$.
        \end{solution}
        \vfill
        \part[3]
        What are the eigenvalues for $A^2-2A$?
        \begin{solution}
            $0,3$
        \end{solution}
        \vfill
    \end{parts}

    \newpage
    % Question 6
    \question
    
    Let $T(x) = Ax$, where $A$ is as defined in Question 5. Let
    $u_1,u_2,u_3,u_4$ also be as defined in Question 5.
    \begin{parts}
        \part[4]
        Give two vectors $v,w$ such that the triangle with vertices
        $\{T(0),T(v),T(w)\}$ has 6 times the area as the triangle with vertices
        $\{0,v,w\}$. Be sure to justify your answer. (Hint: It is unnecessary
        to compute the area of these triangles.)
        \begin{solution}
            Let $v=u_2$ and $w=u_4$. Then $T(u_2)=2u_2$ and $T(u_4)=3u_4$. So
            the area of the triangle increased by a factor of 6.
        \end{solution}
        \vfill
        \part[4]
        Find a basis for each of the following subspaces. If a subspace is
        trivial, then write $\emptyset$ for its basis.
        \begin{itemize}
            \item 
                $\nll(A-2I)$
                \begin{solution}
                    This is a basis for the eigenspace corresponding to 2,
                    $\{u_2,u_3\}$.
                \end{solution}
                \vfill
            \item
                $\nll(A^2-3I)$.
                \begin{solution}
                    Since $3$ is not an eigenvalue of $A^2$, this subspace is
                    trivial so a basis is $\emptyset$.
                \end{solution}
                \vfill
        \end{itemize}
        \part[4] 
        Let $B=\{u_1,u_2,u_3,u_4\}$ be a basis.
        \begin{itemize}
            \item 
                What is the general solution to $Ax=u_2+2u_3$?
                \begin{solution}
                    \[
                        x=(1/2u_2+u_3)+s_1(u_1)
                    \]
                \end{solution}
                \vfill
            \item
                Let $y$ be a particular solution to the above linear system.
                What is $[y]_B$?
                \begin{solution}
                    \[
                        (0,1/2,1,0)
                    \]
                \end{solution}
                \vfill
        \end{itemize}
    \end{parts}
\end{questions}
\end{document}
